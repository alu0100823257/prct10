
\documentclass[12pt,a4paper]{article}
\usepackage{latexsym,amsfonts,amssymb,amstext,amsthm,float,amsmath}
\usepackage[spanish]{babel}
\usepackage[latin1]{inputenc}
\usepackage[dvips]{epsfig}
\usepackage{doc}
\begin{document}
\begin{figure}
\begin{center}
\includegraphics[scale=0.1]{src/imagen1.eps}
\end{center}
\end{figure}
\author{Shaila Verona Rodriguez}
\title{aproximacion de el numero $\pi$}
\date{Viernes 11 de Abril del 2014}
\maketitle 
\tableofcontens
\begin{abstract}
El objetivo de este informe es tratar de explicar el numero Pi en un documento \LaTeX{}~\cite{Latexinf}
\end{abstract}
\section{definicion}
π (pi) es la relación entre la longitud de una circunferencia y su diámetro, en geometría euclidiana. Es un número irracional y una de las constantes matemáticas más importantes. Se emplea frecuentemente en matemáticas, física e ingeniería. El valor numérico de π, truncado a sus primeras cifras, es el siguiente:

    \pi \approx 3,14159265358979323846 \; \dots 

El valor de π se ha obtenido con diversas aproximaciones a lo largo de la historia, siendo una de las constantes matemáticas que más aparece en las ecuaciones de la física, junto con el número e. Cabe destacar que el cociente entre la longitud de cualquier circunferencia y la de su diámetro no es constante en geometrías no euclídeas.
También como hemos visto en prácticas anteriores, el numero pi se puede calcular mediante integración:

$$\int_{0}^{1} \! \frac{4}{1+x^2}\, dx = 4(atan(1) -atan(0)) = \pi $$
\section{Usos del numero \pi en las matemáticas}
El numero \pi~\cite{PI} es usado en varios campos de las matemáticas como por ejemplo:
\begin{enumerate}
\item
En geometría
\item
En calculo
\item 
En probabilidad
\item
En analisis matemático
\end {enumerate}
\section{Mayor numero de decimales}
En la época actual el mayor numero de decimales obtenido se llevó a cabo por Shigeru Kondo, obteniendo 10.000.000.000.000 cifras.
\begin{table}{}
\begin{tabular}{lrcl}
Año & Nombre & Ordenador & Número de decimales \\ \hline
1949 & Reitwiesner & ENIAC & 2.037 \\ \hline
1959 & Guilloud & IBM 704 & 16.167 \\ \hline
1986 & Bailey & CRAY-2 & 29.360.111 \\ \hline
2011 & Kondo & & 10.000.000.000.000 \\ \hline
\end{tabular}
\end {table}
section{Bibliografía}
\begin{itemize}
  \bibitem{PI}
  es.wikipedia.org/wiki/Número\_\pi \hline
  \bibitem{Latexinf}
     www.juegosdelogica.com/numero\_\pi.htm
\end{itemize}
\end{document}


